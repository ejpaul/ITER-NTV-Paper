\documentclass[aip, pop, preprint]{revtex4-1}

\usepackage{graphicx}%
\usepackage{dcolumn}%
\usepackage{bm}%

\usepackage[utf8]{inputenc}
\usepackage{amssymb}
\usepackage{placeins}
\usepackage{amsmath}
\usepackage{amsthm}
\usepackage{commath}
\usepackage[citecolor=blue,colorlinks=true,linkcolor=blue]{hyperref}
\usepackage[margin=1in]{geometry}

\renewcommand{\thefigure}{\thesection.\arabic{figure}}
\renewcommand{\theequation}{\thesection.\arabic{equation}}
\numberwithin{figure}{section}
\numberwithin{equation}{section}

\newcommand{\partder}[2]{\dfrac{\partial  #1}{\partial  #2}} % partial der.
\newcommand{\der}[2]{\dfrac{d #1}{d  #2}}

\begin{document}
\title{Rotation and Neoclassical Ripple Transport in ITER}
\author{E. J. Paul}
\affiliation{Department of Physics, University of Maryland, College Park, MD 20742, USA}
\email{ejpaul@umd.edu}

\author{M. Landreman}
\affiliation{Institute for Research in Electronics and Applied Physics, University of Maryland, College Park, MD 20742, USA}
\email{mattland@umd.edu}

\author{W. Dorland}
\affiliation{Department of Physics, University of Maryland, College Park, MD 20742, USA}
\email{bdorland@umd.edu}

\author{F. M. Poli}
\affiliation{Princeton Plasma Physics Laboratory, Princeton, NJ 08543, USA}
\email{fpoli@pppl.gov}

\author{D. A. Spong}
\affiliation{Oak Ridge National Laboratory, Oak Ridge, TN 37831, USA}
\email{spongda@ornl.gov}

\author{H. M. Smith}
\affiliation{Max-Planck-Institut f\"{u}r Plasmaphysik, 17491 Greifswald, Germany}
\email{hakan.smith@ipp.mpg.de}

\begin{abstract}

Neoclassical transport in the presence of non-axisymmetric magnetic fields causes a toroidal torque known as neoclassical toroidal viscosity (NTV). The toroidal symmetry of ITER will be broken by the finite number of toroidal field coils and by test blanket modules (TBMs). The addition of ferritic inserts (FIs) will decrease the magnitude of the toroidal field ripple. 3D magnetic equilibria in the presence of toroidal field ripple and ferromagnetic structures are calculated for an ITER steady-state scenario using the Variational Moments Equilibrium Code (VMEC). Neoclassical transport quantities in the presence of these error fields are calculated using the Stellarator Fokker-Planck Iterative Neoclassical Conservative Solver (SFINCS). These calculations fully account for $E_r$, flux surface shaping, multiple species, and collisionality rather than applying analytic NTV formulae. As NTV is a complicated nonlinear function of $E_r$, we study its behavior over a plausible range of $E_r$. We estimate the toroidal flow, and hence $E_r$, using a semi-analytic turbulent intrinsic rotation model and NUBEAM calculations of neutral beam torque. The magnitude of NTV torque density at large radii ($r/a \gtrsim$ 0.6) is comparable to the NBI torque density at small radii ($r/a \lesssim$ 0.4), but is opposite in direction and may significantly damp rotation in the edge. The NTV from the $n = 18$ perturbation dominates that from lower $n$ perturbations of the TBMs. 
\end{abstract}

\maketitle

\section{Introduction}

Toroidal rotation is critical to the experimental control of tokamaks: the magnitude of rotation is known to suppress resistive wall modes,\cite{Bondeson1994, Garofalo2002} while rotation shear can decrease microinstabilities and promote formation of transport barriers.\cite{Burrell1997, Terry2000} As some ITER scenarios will be above the no-wall stability limit,\cite{Liu2004} it is important to understand the sources and sinks of angular momentum for stabilization of external kink modes. One such sink (or possibly source) is the toroidal torque caused by 3D non-resonant error fields, known as neoclassical toroidal viscosity (NTV). Dedicated NTV experiments have been conducted in the Mega Amp Spherical Tokamak (MAST),\cite{Hua2010} the Joint European Tokamak (JET),\cite{Lazzaro2002, DeVries2008b} Alcator C-MOD,\cite{Wolfe2005}  DIII-D,\cite{Garofalo2008,Reimerdes2009} JT-60U,\cite{Honda2014} and the National Spherical Tokamak Experiment (NSTX).\cite{Zhu2006} 

In addition to the ripple due to the finite number (18) of toroidal field (TF) coils, the magnetic field in ITER will be perturbed by ferromagnetic components including ferritic inserts (FIs) and test blanket modules (TBMs). TBMs will be installed in three equatorial ports to test tritium breeding and extraction of heat from the blanket. The structural material for these modules is ferritic steel and will produce additional error fields in response to the background field. The TBMs will be installed during the H/He phase in order to test their performance in addition to their possible effects on confinement and transport.\cite{Chuyanov2010} It is important to understand their effect on transport during the early phases of ITER, including their influence on angular momentum transport. Experiments at DIII-D using mock-ups of TBMs found a reduction in toroidal rotation by as much as 60\% due to an $n = 1$ locked mode.\cite{Schaffer2011} Here $n$ is the toroidal mode number. Compensation by $n=1$ control coils enable access to low NBI torque (1.1 Nm) regimes relevant to ITER without rotation collapse.\cite{Lanctot2017} In addition to TBMs, ferritic steel plates will be installed in each of the toroidal field coil sections in order to mitigate energetic particle loss due to TF ripple.\cite{Tobita2003} As FIs will decrease toroidal field ripple, they may decrease the NTV in ITER. Experiments including FIs on JT-60U\cite{Urano2007} and JFT-2M\cite{Kawashima2001} have found a reduction in counter-current rotation with the addition of FIs.

While  the bounce-averaged radial drift vanishes in a tokamak, trapped particles may wander off the flux surface in the presence of non-axisymmetric error fields. Particles trapped poloidally can drift radially as the parallel adiabatic invariant, $J_{||} = \oint v_{||} dl$, becomes a function of toroidal angle in broken symmetry. Here $v_{||}$ is the velocity coordinate along $\bm{b} = \bm{B}/B$ and integration is taken along the field between bounce points. If local ripple wells exist along a field line and the collisionality is small enough that helically trapped particles can complete their collisionless orbits, these trapped particles may grad-$B$ drift away from the flux surface.\cite{Stringer1972} For a general electric field, the electron and ion fluxes are not necessarily identical. As ambipolarity must be restored for charge conservation, a radial electric field develops to hold the ions back. The radial current caused by outward ion flow induces a $\bm{J} \times \bm{B}$ torque which is typically counter-current. 

Analytic expressions for neoclassical fluxes in several rippled tokamak regimes have been derived, making assumptions about the magnitude of the perturbing field, electric field, magnetic geometry, collisionality, and the collision operator. Often multiple regimes are needed to describe all radial positions, classes of particles, and helicities of the magnetic field for a single discharge. When collisions set the radial step size of trapped particles, the transport scales as $1/\nu$ where $\nu$ is the collision frequency. The $1/\nu$ regime can be relevant for both ripple trapped and banana particles. When the effective collision frequency is smaller than the toroidal drift frequency, fluxes from boundary layers become important. Transport from the collisional trapped-passing boundary layer leads to fluxes that scale as $\sqrt{\nu}$ and $\nu$. When the collisionality is sufficiently low, the collisionless detrapping/retrapping layer becomes significant, where fluxes scale as $\nu$. Banana particles can become passing particles due to the variation of $B_{\max}$ along their drift trajectory,\cite{Shaing2009} and ripple trapped particles can also experience collisionless detrapping from ripple wells to become toroidally trapped.\cite{Shaing1982a, Shaing1982b} If the collisionality is small compared with the typical toroidal precession frequency of trapped particles, the resonant velocity space layer where the bounce-averaged toroidal drift vanishes can experience significant radial drifts, leading to superbanana-plateau transport.\cite{Shaing2009_sbp} In the presence of a strong radial electric field, the bounce-harmonic resonance between the parallel bounce motion and drift motion of trapped particles can also result in enhanced transport.\cite{Linsker1982,Park2009} The $1/\nu$ and $\sqrt{\nu}$ stellarator regimes for helically-trapped particles have been formulated by Galeev and Sagdeev,\cite{Galeev1969} Ho and Kulsrud,\cite{Ho1987} and Frieman.\cite{Frieman1970} These results were generalized to rippled tokamaks in the $1/\nu$ regime by Stringer \cite{Stringer1972} and Connor and Hastie\cite{Connor1973} and in the $\sqrt{\nu}$ regime by Kovrizhnykh.\cite{Kovrizhnykh1984} Linsker and Boozer \cite{Linsker1982} evaluated banana diffusion particle and heat transport in the $1/\nu$ and $\nu$ regimes, and Davidson\cite{Davidson1976} evaluated banana diffusion $1/\nu$ transport. Shaing has formulated the theory for NTV torque due to banana diffusion in the $1/\nu$,\cite{Shaing2003} $\nu-\sqrt{\nu}$,\cite{Shaing2008} $\nu$,\cite{Shaing2009} and superbanana-plateau \cite{Shaing2009_sbp} regimes in addition to an approximate analytic formula which connects these regimes.\cite{Shaing2010}

%when ripple-trapped orbits are collisionless, $\nu_* \ll (\delta_B/\epsilon)^{3/2}$.\cite{Shaing2003}
%can lead to increased radial step sizes in the superbanana-plateau regime. 
%Passing particles can contribute to NTV near resonant surfaces.\cite{Satake2013} We find no concentration of NTV torque density near low-order rational surface which suggests non-resonant braking is dominant. Particles trapped in local ripple wells can experience a net radial drift due to their . Additionally, trapped particles executing banana orbits can  Particles trapped in local ripple wells and in the poloidal variation of the field can experience a net radial drift and contribute to radial current. When the radial excursions of trapped particles are limited by collisions the transport scales as $1/\nu$, and when drift orbits are limited by $E \times B$ precession the transport scales with $\nu$. The $\sqrt{\nu}$ regime describes the collisional boundary layer in the transition between these two regimes. We will also consider several resonant regimes. At very small $E_r$ the resonance between the $E \times B$ and magnetic drift frequencies can result in the superbanana-plateau regime for sufficiently small collisionality, and at large $E_r$ a bounce-harmonic resonance between the parallel bounce motion and drift motion of trapped particles can result in enhanced transport.\cite{Linsker1982,Park2009} The $1/\nu$ and $\sqrt{\nu}$ regimes for stellarator transport due to helical ripple have been formulated by Galeev and Sagdeev,\cite{Galeev1969} Ho and Kulsrud,\cite{Ho1987} and Frieman.\cite{Frieman1970} These results were generalized to the limit of ripple magnitude ($\delta_B$) much smaller than the inverse aspect ratio ($\epsilon$) by Stringer \cite{Stringer1972} and Connor and Hastie\cite{Connor1973} for tokamak ripple trapping in the $1/\nu$ regime, by Kovrizhnykh\cite{Kovrizhnykh1984} in the $\sqrt{\nu}$ regime, and by Shaing\cite{Shaing1982a, Shaing1982b} in the $\sqrt{\nu}$ and $\nu$ regimes. Linsker and Boozer \cite{Linsker1982} evaluated banana diffusion particle and heat transport in the $1/\nu$ and $\nu$ regimes, and Davidson\cite{Davidson1976} evaluated banana diffusion $1/\nu$ transport. Shaing has formulated the theory for NTV torque due to banana diffusion in the $1/\nu$,\cite{Shaing2003} $\nu$,\cite{Shaing2009} $\sqrt{\nu}$,\cite{Shaing2008} and superbanana-plateau \cite{Shaing2009_sbp} regimes. 

The calculation of NTV torque requires two steps: (i) determine the equilibrium magnetic field in the presence of ripple and (ii) solve a drift kinetic equation (DKE) with the magnetohydrodynamic (MHD) equilibrium or apply reduced analytic formulae. The first step can be performed using various levels of approximation. The simplest method is to superimpose 3D ripple vacuum fields on an axisymmetric equilibrium, ignoring the plasma response.  A second level of approximation is to use a linearized 3D equilibrium code such as the Ideal Perturbed Equilibrium Code (IPEC)\cite{Park2009} or linear MD3-C1.\cite{Jardin2008} A third level of approximation is to solve nonlinear MHD force balance using a code such as the Variational Moments Equilibrium Code (VMEC)\cite{Hirshman1986a} or M3D-C1\cite{Ferraro2010} run in nonlinear mode. In this paper we use free-boundary VMEC to find the MHD equilibrium in the presence of toroidal field ripple, FIs, and TBMs. 

Many previous NTV calculations\cite{Zhu2006,Hua2010,Cole2011,Park2009} have been performed using reduced analytic models, although a DKE must be solved numerically in order to avoid severe assumptions. Solutions of the bounce-averaged kinetic equation have been found to agree with Shaing's analytic theory except in the transition between regimes.\cite{Sun2010} However, the bounce-averaged kinetic equation does not include contributions from bounce and transit resonances. Discrepancies have been found between numerical evaluation of NTV using the Monte Carlo neoclassical solver FORTEC-3D and analytic formulae for the $1/\nu$ and superbanana-plateau regimes.\cite{Satake2011a,Satake2011b} The quasilinear NEO-2 has been found to differ from Shaing's connected formulae,\cite{Shaing2010} especially in the edge where the large aspect ratio assumption breaks down.\cite{Martitsch2016} Rather than applying such reduced models, in this paper a DKE is solved using the Stellarator Fokker-Planck Iterative Neoclassical Conservative Solver (SFINCS) \cite{Landreman2014} to calculate neoclassical particle and heat fluxes for an ITER steady-state scenario. The SFINCS code does not exploit any expansions in collisionality, size of perturbing field, or magnitude of the radial electric filed. It also allows for realistic experimental magnetic geometry rather than using simplified flux surface shapes. All trapped particle effects including ripple-trapping,\cite{Stringer1972} banana diffusion,\cite{Linsker1982} and bounce-resonance\cite{Linsker1982} are accounted for in these calculations. The DKE solved by SFINCS ensures intrinsic ambipolarity for axisymmetric or quasisymmetric flux surfaces in the presence of a radial electric field while this property is not satisfied by other codes such as DKES.\cite{Hirshman1986b,Rij1989} This prevents spurious NTV torque density, which is proportional to the radial current. As SFINCS makes no assumption about the size of ripple, it can account for nonlinear transport due to locally trapped particles. The deviation from the quasilinear assumption has been found to be significant in benchmarks between SFINCS and NEO-2.\cite{Martitsch2016}

%Ripple trapping has been shown to provide relatively small contribution to neoclassical transport in comparison to banana drift transport unless the ripple is large in comparison to the inverse aspect ratio, $\epsilon $.\cite{Garbet2010, Calvo2015}

In addition to NTV, neutral beams will provide an angular momentum source for ITER. As NBI torque scales as $P/E^{1/2}$ for input power $P$ and particle energy $E$, ITER's neutral beams, with $E = 1$ MeV and $P = 33$ MW, will provide less momentum than in other tokamaks such as JET, with $E = 125$ keV for $P = 34$ MW.\cite{Ciric2011} NBI-driven rotation will also be smaller in ITER because of its relatively large moment of inertia, with major radius $R = 6$ m compared to 3 m for JET. 

However, spontaneous rotation may be significant in ITER. Turbulence can drive significant flows in the absence of external momentum injection, known as intrinsic or spontaneous rotation. This can be understood as a turbulent redistribution of toroidal angular momentum to produce large directed flows. For perturbed tokamaks this must be in the approximate symmetry direction. According to gyrokinetic orderings and inter-machine comparisons by Parra,\cite{Parra2012} intrinsic toroidal rotation scales as $V_{\zeta} \sim  T_i/I_p$ where $T_i$ is the ion temperature and $I_p$ is the plasma current, and core rotations may be on the order of 100 km/s (ion sonic Mach number $M_i \approx 8\%$) in ITER. Scalings with $\beta_N = \beta_T a B_T/I_P$, where $\beta_T =  2\mu_0 P/B_T^2$, $B_T$ is the toroidal magnetic field in tesla, $a$ is the minor radius in meters, and $P$ is the plasma pressure, by Rice\cite{Rice2007} predict rotations of a similar scale, $M_i \approx 30\%$ or $V_{\zeta} \approx 400$ km/s. Co-current toroidal rotation appears to be a common feature of H-mode plasmas and has been observed in electron cyclotron heated (ECH),\cite{DeGrassie2007} ohmic,\cite{DeGrassie2007} and ion cyclotron range of frequencies (ICRF) \cite{Noterdaeme2003} heated plasmas. Gyrokinetic GS2 simulations have also shown that low collisionality tokamaks have an inward radial momentum flux, corresponding to a rotation profile peaked in the core toward the co-current direction.\cite{Barnes2013} In an up-down symmetric tokamak, radial intrinsic angular momentum flux can be shown to vanish to lowest order in $\rho_* = \rho_i/a$, where $\rho_i = v_{ti}m_i c/{eB}$ is the gyroradius and $v_{ti} = \sqrt{2T_i/m_i}$ is the ion thermal velocity, but neoclassical departures from an equilibrium Maxwellian can break this symmetry and cause non-zero rotation in the absence of input momentum.\cite{Barnes2013} 

In section \ref{steadystate} the ITER steady state scenario considered is discussed. In section \ref{vmec} free boundary MHD equilibrium in the presence of field ripple are presented. In section \ref{rotation} we estimate rotation driven by NBI and turbulence. This flow velocity is related to $E_r$ in section \ref{Erandv}. The influence of TF ripple, TBMs, and FIs on neoclassical transport is evaluated, and a radial profile of torque is presented in section \ref{torque}. In section \ref{scaling} the scaling of transport calculated with SFINCS is compared with that predicted by NTV theory, and in section \ref{heatflux} neoclassical heat fluxes in the presence of ripple are presented. In section \ref{mds}, we assess several tangential magnetic drift models on the transport for this ITER scenario. In section \ref{summary} we summarize the results and conclude.

\section{ITER Steady State Scenario}\label{steadystate}

\FloatBarrier

\begin{figure}[h!]
\centering
\includegraphics[width=0.7\textwidth]{profiles.png}
\caption{\label{fig:profiles} Radial profiles of temperature, density, and safety factor for the ITER steady state scenario.\cite{Poli2014} Black dashed lines indicate the radial locations that will be considered for neoclassical calculations.}
\end{figure}

We consider an advanced ITER steady state scenario with significant bootstrap current and reversed magnetic shear.\cite{Poli2014} The input power includes 33 MW NBI, 20 MW electron cyclotron (EC), and 20 MW lower hybrid (LH) heating for a fusion gain of $Q = 5$ and toroidal current of 9 MA. The discharge was simulated using the Tokamak Simulation Code (TSC) \cite{Jardin1986} and TRANSP \cite{Hawryluk1980} using a Coppi-Tang \cite{Jardin1993} transport model and EPED1 \cite{Snyder2011} pedestal modeling. The density, temperature, and safety factor $q$ profiles are shown in figure \ref{fig:profiles}. For neoclassical calculations we consider a three species plasma (D, T, and electrons), and we assume that $n_D = n_T = n_e/2$. Neoclassical transport will be analyzed in detail at the radial locations indicated by dashed horizontal lines ($r/a = 0.5, 0.7, 0.9$). Throughout we will use the radial coordinate $r/a \propto \sqrt{\Psi_{\text{T}}}$ where $\Psi_{\text{T}}$ is the toroidal flux.

\FloatBarrier

\section{Free Boundary Equilibrium Calculations and Ripple Magnitude} \label{vmec}

The magnetic equilibrium was computed using the density, temperature, and $q$ profiles from TRANSP along with filamentary models of the toroidal field (TF), poloidal field (PF), and center stack (CS) coils and their corresponding currents. The vacuum fields produced by the three TBMs and the FIs have been modeled using FEMAG.\cite{Shinohara2009} The VMEC free-boundary equilibrium \cite{Hirshman1986a} is computed for four geometries: (i) including only the TF ripple, (ii) including TF ripple, TBMs, and FIs, (iii) TF ripple and FIs, and (iv) axisymmetric geometry.  

We define the magnitude of the magnetic field ripple to be,
\begin{gather}
\delta_B = (B_{\text{max}}-B_{\text{min}})/(B_{\text{max}} + B_{\text{min}}), 
\end{gather}
where the maximum and minimum are evaluated at fixed radius and poloidal angle $\theta$. In figure \ref{fig:ripplecontour}, $\delta_B$ is plotted on the poloidal plane for the three rippled VMEC equilibria. A fourth case is also shown in which the component of $\bm{B}$ with $\abs{n} = 18$ was removed from the geometry with TBMs and FIs in order to demonstrate the magnitude of the ripple produced by the TBMs alone (bottom right). When only TF ripple is present, significant ripple persists over the entire outboard side, while in the configurations with FIs the ripple is much more localized in $\theta$. When TBMs are present, the ripple is higher in magnitude near the outboard midplane ($\delta_B \approx 1.4\%$), while in the other magnetic configurations $\delta_B \approx$ 1\% near the outboard midplane. For comparison, the TF ripple during standard operations is $0.08\%$ in JET \cite{DeVries2008b} and $0.6\%$ in ASDEX Upgrade.\cite{Martitsch2016} 

In figure \ref{fig:toroidalripple}, the magnitude of $\bm{B}$ is plotted as a function of toroidal angle $\zeta$ at $\theta = 0$ and $\theta = \pi/4$. Away from the midplane ($\theta = \pi/4$) the FIs greatly decrease the magnitude of the TF ripple. Near the midplane the FIs do not decrease the magnitude of the toroidal ripple as strongly, as the number of steel plates is reduced near the midplane.\cite{Shinohara2009} The ferromagnetic steel of the TBMs concentrates magnetic flux and locally decreases $B$ in the plasma near its location. This adds an enhancement of $\delta_B$ near $\theta = 0$. 
\FloatBarrier

\begin{figure}[h!]
\centering
\includegraphics[width=0.7\textwidth]{ripplecontour.png}
\caption{\label{fig:ripplecontour} Magnetic field ripple, $\delta_B = (B_{\text{max}}-B_{\text{min}})/(B_{\text{max}} + B_{\text{min}})$, is plotted on the poloidal plane for VMEC free boundary equilibria including (i) only TF ripple (top left), (ii) TF ripple, TBMs, and FIs (top right), (iii) TF ripple and FIs (bottom left), and (iv) with TBMs only (bottom right). FIs decrease the poloidal extent of the ripple, while TBMs add an additional ripple near the outboard midplane.}
\end{figure}

\begin{figure}[h!]
\centering
\includegraphics[width=0.7\textwidth]{toroidalripple.png}
\caption{\label{fig:toroidalripple} The magnitude of $\bm{B}$ as a function of toroidal angle ($\zeta$) at $r/a = 1$, $\theta = 0$ and $\pi/4$. Vertical dashed lines indicate the toroidal locations of the TBM ports. The mitigating effect of the FIs is stronger away from the midplane, where an increased number of steel plates are inserted. The TBMs add an additional ripple near their locations at $\theta = 0$. }
\end{figure}

\FloatBarrier

\section{Estimating Toroidal Rotation}\label{rotation}

In order to predict the ripple transport in ITER, the radial electric field, $E_r = - \Phi'(r) $, must be estimated, as particle and heat fluxes are nonlinear functions of $E_r$. This is equivalent to predicting the parallel flow velocity, $V_{||}$, which scales monotonically with $E_r$.  As we simply wish to determine a plausible value of $E_r$, the difference between $V_{||}$ and $V_{\zeta} = \Omega_{\zeta}(r) \langle R \rangle$, the toroidal flow, will be unimportant for our estimates. Here the flux surface average is denoted by $\langle ... \rangle$,
\begin{gather}
\langle ... \rangle = \frac{1}{V'} \int_0^{2 \pi} d \theta \int_0^{2 \pi} d \zeta \sqrt{g} (...)
\\ V' = \int_0^{2\pi} d \theta \int_0^{2 \pi} d \zeta \sqrt{g},
\end{gather}
where $\sqrt{g}$ is the Jacobian. As $B$ and $I_P$ will be parallel in ITER, $V_{\zeta}$ and $V_{||}$ point in the same direction.  In our calculation of $V_{\zeta}$ the neoclassical torque is ignored. Instead $E_r$ is viewed as an input to neoclassical calculations from which the NTV torque can be obtained. For this rotation calculation, angular momentum transport due to neutral beams and turbulence will be considered. There is an additional torque caused by the radial current of orbit-lost alphas,\cite{Rosenbluth1996} but it will be negligible ($\approx 0.006$ Nm/m$^3$). The following time-independent momentum balance equation is considered in determining $\Omega_{\zeta}(r)$,
\begin{gather}
\nabla \cdot \Pi_{\zeta}^{\text{turb}}(\Omega_{\zeta}) + \nabla \cdot \Pi_{\zeta}^{\text{NC}}(\Omega_{\zeta}) = \tau^{\text{NBI}},
\end{gather}
where $\Pi^{\text{turb}}_{\zeta}$ and $\Pi^{\text{NC}}_{\zeta}$ are the toroidal angular momentum flux densities due to turbulent and neoclassical transport and $\tau^{\text{NBI}}$ is the NBI torque density. For this paper the feedback of $\Pi_{\zeta}^{\text{NC}}$ on $\Omega_{\zeta}$ will not be calculated. Determining the change in rotation due to NTV would require iteratively solving this equation for $\Omega_{\zeta}$, as $\Pi_{\zeta}^{\text{NC}}$ is a nonlinear functions of $\Omega_{\zeta}$. 

The quantity $\Pi_{\zeta}^{\text{turb}}$ consists of a diffusive term as well as a term independent of $\Omega_{\zeta}$ which accounts for turbulent intrinsic rotation. An angular momentum pinch will not be considered for this analysis. 
\begin{gather}
\Pi_{\zeta}^{\text{turb}} = -m_i n_i \chi_{\zeta} \langle R^2 \rangle \partder{\Omega_{\zeta}}{r} + \Pi_{\text{int}}
\end{gather}
Here $m_i$ is the ion mass, $n_i$ is the ion density, and $\chi_{\zeta}$ is the toroidal ion angular momentum diffusivity. Ignoring NTV torque, we will solve the following angular momentum balance equation,
\begin{gather}
m_i \frac{1}{V'} \partder{}{r} \left( V' n_i \chi_{\zeta} \langle R^2 \rangle \partder{\Omega_{\zeta}}{r} \right) =  -\frac{1}{V'} \partder{}{r} \left( V' \Pi_{\text{int}} \right) + \tau^{\text{NBI}}.
\label{eq:angularmomentum}
\end{gather}
Eq. \ref{eq:angularmomentum} is is a linear inhomogeneous equation for $\Omega_{\zeta}$, as the right hand side is independent of $\Omega_{\zeta}$. We can therefore solve for the rotation due to each of the source terms individually and add the results to obtain the rotation due to both NBI torque and turbulent intrinsic torque. 

The NBI-driven rotation profile is evolved by TRANSP assuming $\chi_{\zeta} = \chi_{i}$, the ion heat diffusivity. The total beam torque density, $\tau^{\text{NBI}}$, is calculated by NUBEAM including collisional, $\bm{J} \times \bm{B}$, thermalization, and recombination torques. The following momentum balance equation is solved to compute $\Omega_{\zeta}$ driven by NBI,
\begin{gather}
\tau^{\text{NBI}} = -\frac{1}{V'} \partder{}{r} \left( V' m_i n_i \chi_{i} \langle R^2 \rangle \partder{\Omega_{\zeta}}{r} \right).
\end{gather} 

We consider a semi-analytic intrinsic rotation model to determine the turbulent-driven rotation,\cite{Hillesheim2015}
\begin{gather}
\Omega_{\zeta}(r) = - \int_{r}^a \frac{v_{ti} \rho_{*,\theta}} {2 P_r L_T^2} \widetilde{\Pi} (\nu_*) \, d r',
\label{eq:Hillesheim}
\end{gather} 
where $\rho_{*,\theta} = v_{ti} m_i/(e B_{\theta} \langle R \rangle)$ is the poloidal normalized gyroradius, and the Prandtl number $P_r = \chi_{\zeta}/\chi_i$ is taken to be 1. The normalized collision frequency is $\nu_* = q R v_{ti}/(\nu_{ii} \epsilon^{3/2})$ where $\epsilon = r/\langle R \rangle$ is the inverse aspect ratio and $L_T = - \left( \partial \ln T_i/ \partial r \right)^{-1}$ is the temperature gradient scale length. Equation \ref{eq:Hillesheim} is obtained assuming that $\Pi_{\text{int}}$ balances turbulent momentum diffusion in steady state, $\Pi_{\text{int}} = m_i n_i \chi_{\zeta} \langle R^2 \rangle \partial \Omega_{\zeta}/\partial r$. This model considers the intrinsic torque driven by the neoclassical diamagnetic flows, such that $\Omega_{\zeta} \sim \rho_{*,\theta} v_{ti}/\langle R \rangle$ and $\Omega_{\zeta} \Pi_{\text{int}}/Q \sim \rho_{*, \theta}$. It is also assumed that $\Omega_{\zeta} = 0$ at the wall. 

The quantity $\widetilde{\Pi} (\nu_*)$ is an order unity function which characterizes the collisionality dependence of rotation reversals, determined from gyrokinetic turbulence simulations,\cite{Barnes2013}
\begin{gather}
\widetilde{\Pi} (\nu_*) = \frac{(\nu_*/\nu_c -1)}{1 + (\nu_*/\nu_c)},
\end{gather}
where $\nu_c = 1.7$. Because of ITER's low collisionality, we do not expect a rotation reversal, which is correlated with transitioning between the banana and plateau regimes. Equation \ref{eq:Hillesheim} was integrated using profiles for the ITER steady state scenario. 

The rotation predicted by these models is shown in figure \ref{fig:rotation_estimate}. NBI torque contributes to significant rotation at $r/a \lesssim 0.4$ where the torque density also peaks (see figure \ref{fig:alltorque}), while turbulent torque produces rotation in the pedestal due to the $L_T^{-2}$ scaling of our model.  The intrinsic rotation calculated is comparable to that predicted from theoretical scaling arguments by Parra,\cite{Parra2012} $V_{\zeta} \approx 100$ km/s. At the radii that will be considered for neoclassical calculations (indicated by dashed vertical lines), intrinsic turbulent rotation may dominate over that due to NBI. However, we emphasize that is is an estimate based on scaling arguments, as much uncertainty is inherent in predicting turbulent rotation. 

\FloatBarrier

\begin{figure}[h!]
\centering
\includegraphics[width=0.7\textwidth]{rotationestimate.png}
\caption{\label{fig:rotation_estimate} Toroidal rotation $V_{\zeta}$ due to turbulence and NBI (top) is shown along with  corresponding Alfv\`{e}n Mach number (bottom, solid), and ion sonic Mach number (bottom, bulleted). The intrinsic rotation calculation uses a semi-analytic model of turbulent momentum redistribution.\cite{Hillesheim2015} The NBI rotation is calculated from turbulent diffusion of NBI torque using NUBEAM and TRANSP.\cite{Poli2014} Dashed vertical lines indicate the radial positions where SFINCS calculations are performed. }
\end{figure}

For stabilization of the resistive wall mode (RWM) in ITER, it has been estimated that a critical central rotation frequency $ \Omega_{\zeta}(0) \gtrsim 5\%$ of the Alfv\`{e}n frequency, $\omega_A = B/(\langle R\rangle\sqrt{\mu_0 \rho_i})$, must be achieved given a peaked rotation profile.\cite{Liu2004} With a central rotation of $\Omega_{\zeta}(0) \approx 2\% \, \omega_A$ in figure \ref{fig:rotation_estimate}, it may be difficult to suppress the RWM in ITER with rotation alone. As this calculation does not take into account magnetic braking, $\Omega_{\zeta}(0)/\omega_A$ is likely to be smaller than what is shown. Additionally, the TBM are known to increase the critical rotation frequency as they have a much shorter resistive time scale than the wall.\cite{Liu2004} More recent analysis has shown that even above such a critical rotation value, the plasma can become unstable due to resonances between the drift frequency and bounce frequency.\cite{Berkery2010, Liu2009}

\FloatBarrier

\section{Relationship Between $E_r$ and $V_{||}$}\label{Erandv}
Neoclassical theory predicts a specific linear-plus-offset relationship between $V_{||}$ and $E_r$, but it does not predict a particular value for either $V_{||}$ or $E_r$ in the presence of symmetry-breaking. Neoclassical calculations of $V_{||}$ are made in order to determine an $E_r$ profile consistent with the our estimate of $V_{\zeta} \approx \langle V_{||} B \rangle/\langle B^2 \rangle^{1/2}$ made in section \ref{rotation}. The parallel flow velocity for species $a$ is computed from the neoclassical distribution function,
\begin{gather}
V^a_{||} = \left(\frac{1}{n_a}\right) \int d^3 v \, v_{||} f_a,
\label{eq:parallelflow}
\end{gather}
which we calculate with the SFINCS \cite{Landreman2014} code. 

SFINCS is used to solve a radially-local DKE for the gyro-averaged distribution function, $f_{a1}$, on a single flux surface including coupling between species. 
\begin{gather}
( v_{||} \bm{b} + \bm{v}_E + \bm{v}_{\text{m}a}) \cdot (\nabla f_{a1})  - C(f_{a1}) = - \bm{v}_{\text{m}a} \cdot \nabla \psi \left( \partder{f_{a0}}{\psi} \right) + \frac{Z_a e v_{||} B \langle E_{||} B \rangle}{T_a \langle B^2 \rangle } f_{a0}
\label{kineticequation}
\end{gather} 
\hspace{-1mm}
Here $a$ indicates species, $f_{a0}$ is an equilibrium Maxwellian, $\psi = \Psi_{\text{T}}/2\pi$, and $C$ is the linearized Fokker-Planck collision operator. Gradients are performed at constant $W = m_a v^2/2 + Z_a e \Phi$ and $\mu = v_{\perp}^2/(2B)$. The $\bm{E} \times \bm{B}$ drift is 
\begin{gather}
\bm{v}_E = \frac{1}{B^2} \bm{B} \times \nabla \Phi
\end{gather} 
and the radial magnetic drift is
\begin{gather}
\bm{v}_{\text{m}a} \cdot \nabla \psi = \frac{1}{2\Omega_a B} \left(v_{||}^2 + \frac{v_{\perp}^2}{2} \right) \bm{b} \times \nabla B \cdot \nabla \psi,
\label{magneticdrift}
\end{gather} 
where $v_{\perp}$ is the velocity coordinate perpendicular to $\bm{b}$. The quantity $\Omega_a = Z_aeB/m_a$ is the gyrofrequency. Transport quantities have been calculated using the steady state scenario ion and electron profiles and VMEC geometry. The second term on the right hand side of eq. \ref{kineticequation} proportional to $E_{||}$ is negligible for this non-inductive scenario with loop voltage $ \approx 10^{-4}$ V. For the calculations presented sections \ref{Erandv}, \ref{torque}, \ref{scaling}, and \ref{heatflux}, $\bm{v}_{\text{m}a} \cdot \nabla f_{a1}$ is not included. The effect of keeping this term is shown to be small in section \ref{mds}.

The relationship between $E_r$ and $\langle V_{||} B \rangle/\langle B^2 \rangle^{1/2}$ for electrons and ions at $r/a = 0.9$ is shown in figure \ref{fig:Er_flow}. Note that only one curve is shown for each species as the addition of ripple fields did not change the dependence of $V_{||}$ on $E_r$ significantly ($\leq 5 \%$). While radial transport of heat and particles changes significantly in the presence of small ripple fields (see sections \ref{torque}, \ref{scaling}, and \ref{heatflux}), the parallel flow is much less sensitive to the perturbing field. It can be shown (see appendix \ref{parallelflow}) that when axisymmetry is broken the component of $f_1$ that contributes to $V_{||}$ is of higher order in $\nu_* \ll 1 $ than the component that contributes to the radial particle fluxes, $\Gamma_{\psi}$, and NTV torque density. 

\begin{figure}[h!]
\centering
\includegraphics[width=.7\textwidth]{Er_flow.png}
\caption{\label{fig:Er_flow} SFINCS calculation of the flux surface averaged parallel flow, $\langle B V_{||} \rangle/\langle B^2 \rangle^{1/2}$, at $r/a = 0.9$ for ions and electrons. The addition of ripple does not change the tokamak neoclassical relationship between $E_r$ and $V_{||}$ by a discernible amount on this scale although the radial particle fluxes, $\Gamma_{\psi}$, are sensitive to the perturbing field.}
\end{figure}

We can write $\langle B V_{||}^a \rangle$ in terms of dimensionless transport coefficients $L_{31}$ and $L_{32}$ and thermodynamic drives,
\begin{gather}
\langle  V_{||}^a  B\rangle = L_{31} \frac{c G}{Z_a e n_a} \left[ \frac{1}{n} \der{(nT)}{\psi_P} + Z_a e \der{\Phi}{\psi_P}  + k_{||} \frac{cG}{Z_a e n_a} \der{T}{\psi_P} \right],
\end{gather}
where $2 \pi \psi_P$ is the poloidal flux,  $G$ is defined through $\bm{B} = \beta \nabla \psi + I(\psi) \nabla \theta + G(\psi) \nabla \zeta$ for Boozer angles $\theta$ and $\zeta$, and $k_{||} = (L_{32}/L_{31} - 5/2)$ is the parallel flow coefficient. The low collisionality, large aspect ratio limit\cite{Hinton1976, Hirshman1981} $k_{||} \approx 1.17$ is often assumed in NTV theory\cite{Callen2011, Sun2011} in relating analytic expressions of torque density to toroidal rotation frequency. The value of $k_{||}$ calculated by SFINCS for ITER parameters varies between 1.6 near the edge and 0.6 near the core. 

\FloatBarrier

\section{Torque Calculation}\label{torque}

The NTV torque density, $\tau^{\text{NTV}}$, is calculated from radial particle fluxes, $\Gamma_{\psi}$, 
\begin{gather}
\Gamma_{\psi,a} = \left \langle \int d^3v (\bm{v}_{\text{m}a} \cdot \nabla \psi) f_a \right \rangle,
\label{eq:particleflux}
\end{gather}
using the flux-friction relation,
\begin{gather}
\tau^{\text{NTV}} = - B^{\theta} \sum_a n_a q_a \Gamma_{\psi, a},
\end{gather}
where $B^{\theta} = \bm{B} \cdot \nabla \theta$ and the summation is performed over species. This expression relates radial particle transport to a toroidal angular momentum source caused by the non-axisymmetric field. This relationship can be derived from action-angle coordinates,\cite{Albert2016} neoclassical moment equations,\cite{Shaing1986} or from the definition of the drift-driven flux.\cite{Shaing2006} 

The calculation of $\tau^{\text{NTV}}$ for three geometries at $r/a = 0.9$ is shown in figure \ref{fig:Torque_ErandV}. Here positive corresponds to the co-current direction. The numerically computed NTV torque is found to vanish in axisymmetric geometry, as expected. Overall, the magnitude of $\tau^{\text{NTV}}$ with only TF ripple is larger than that with the addition of both the FIs and the TBMs.  In figure \ref{fig:Torque_comparingTBMandFI} we show that the $\abs{n} < 18$ TBM  ripple produces much less torque than the $\abs{n} = 18$ component of $B$, so the decrease in $\tau^{\text{NTV}}$ magnitude with both FIs and TBMs can be attributed to the decrease in ripple in the presence of FIs. As will be discussed in section \ref{scaling}, neoclassical ripple transport in most regimes scales positively with $\delta_B$. The addition of FIs significantly decreases the magnitude of $\delta_B$ across most of the outboard side, and as a result the magnitude of $\tau^{NTV}$ is significantly decreased. The dashed vertical line indicates the value of $\langle V_{||} B\rangle/\langle B^2 \rangle^{1/2}$ and $E_r$ predicted from the intrinsic and NBI rotation model. At this value of $E_r$ the presence of ferritic components decreases the magnitude of the torque density by about $75\%$. 

The circle indicates the offset rotation at the ambipolar $E_r$. If no other angular momentum source were present in the system, $\tau^{\text{NTV}}$ would drive the plasma to rotate at this velocity. Although $\tau^{\text{NTV}}$ differs significantly between the two geometries they have similar offset rotation velocities, $V_{\zeta}$ = -10 km/s with TF ripple only and -6 km/s with TBMs and FIs. Note that for $E_r$ greater than this ambipolar value, $\tau^{\text{NTV}}$ is counter-current while neutral beams and turbulence drive rotation in the co-current direction, so $\tau^{\text{NTV}}$ is a damping torque. The NTV torque due to TF ripple only is larger in magnitude than $\tau^{\text{NBI}}$ while that with TBMs and FIs is of similar magnitude (see figure \ref{fig:alltorque}). NTV may also compete with the turbulent torque, $\tau^{\text{turb}} = - \nabla \cdot \Pi_{\text{int}} \approx 0.05$ Nm/m$^3$ at this radius. Therefore, NTV torque may be key in determining the edge rotation in ITER. 

% Discussion of Er = 0 peak and collisionality regimes
Note that the magnitude of $\tau^{\text{NTV}}$ peaks near $E_r = 0$ where $1/\nu$ transport becomes dominant. Although $\nu_*$ is sufficiently small such that the superbanana-plateau regime may be significant, the physics of superbanana formation is not included in these SFINCS calculations which do not include $\bm{v}_{\text{m}} \cdot \nabla f_1$. Superbanana-plateau transport will be considered in section \ref{mds}. While the $1/\nu$ regime is avoided due to the sufficiently large $\abs{E_r}$ estimate at this radius, $1/\nu$ transport may become significant at smaller radii, as will be discussed below. The peak at small $\abs{E_r}$ also corresponds to the region where $1/\nu$ transport of particles trapped in local ripple wells becomes relevant. Because of ITER's low collisionality, $\nu_* \ll (\delta/\epsilon)^{3/4}$, diffusion due to ripple trapping may compete with banana diffusion.\cite{Garbet2010} We have confirmed that the TF ripple causes local wells along the field line, corresponding to $\alpha = \epsilon/(qN\delta_B) < 1$.\cite{Stringer1972} Much NTV literature is based on banana diffusion and ripple trapping in the $1/\nu$ regime,\cite{Stringer1972, Connor1974} which is not applicable for the range of $E_r$ applicable to ITER. At $r/a = 0.9$, the $1/\nu$ regime applies for $\abs{E_r} \lesssim 0.2$ kV/m where the effective collision frequency of trapped particles is larger than the $E \times B$ precession frequency. 

NTV torque is often expressed in terms of a toroidal damping frequency, $\nu_{\zeta}$,
\begin{gather}
\tau^{\text{NTV}} = - \nu_{\zeta} \langle R^2 \rangle m n ( \Omega_{\zeta} - \Omega_{\zeta, \text{offset}}),
\end{gather}
where $\Omega_{\zeta, \text{offset}}$ is the offset rotation frequency. We note that $\tau^{\text{NTV}}$ does appear to scale linearly with $E_r$ (and thus $\Omega_{\zeta}$) for $\abs{E_r} \gtrsim 30$ kV/m. However, $\tau^{NTV}$ is a complicated nonlinear function of $\Omega_{\zeta}$ for $\abs{E_r} \lesssim 30$ kV/m at the transition between collision-limited superbanana-plateau transport and $\sqrt{\nu}-\nu$ transport. 

\FloatBarrier

\begin{figure}[h!]
\centering
\includegraphics[width=0.7\textwidth]{Torque_ErandV.png}
\caption{\label{fig:Torque_ErandV} SFINCS calculation of NTV torque density as a function of $E_r$ and ion $\langle V_{||} B \rangle/\langle B^2 \rangle^{1/2}$ at $r/a = 0.9$ is shown for 3 VMEC geometries: (i) axisymmetric (blue dashed), (ii) with TF ripple only (orange dash-dot), and (iii) TF ripple with FIs and TBMs (green solid). The vertical dashed line indicates the estimate of $E_r$ and $V_{\zeta} \approx \langle V_{||} B \rangle/\langle B^2 \rangle^{1/2}$ based on the intrinsic and NBI rotation model. The circle denotes the offset rotation at $V_{||} \approx -10$ km/s. The magnitude of $\tau^{\text{NTV}}$ at this radius is of similar magnitude to the NBI and turbulent torques but is opposite in direction (see figure \ref{fig:alltorque}).}
\end{figure}

In figure \ref{fig:Torque_eandi} we present the contribution to the NTV torque ($\tau^{NTV}$) density at $r/a = 0.9$ by the electrons and ions in the presence of TF ripple only (left) and TF ripple with ferromagnetic components (right). Note that the $E_r$ corresponding to the offset rotation frequency for the electrons is positive while that of the ions is negative. At the $E_r$ predicted by the intrinsic and NBI rotation model, $\tau^{NTV}$ due to the electron particle flux is positive while that due to ion particle flux is negative. At all radial locations the electron contribution to $\tau^{NTV}$ is less than 10\% of the total torque density. 

\begin{figure}[h!]
\centering
\includegraphics[width=0.7\textwidth]{Torque_eandi.png}
\caption{\label{fig:Torque_eandi} Total (blue dashed), electron (yellow dash-dot), and ion (green solid) contributions to NTV torque density at $r/a = 0.9$ for TF ripple only geometry (left) and TF ripple with ferromagnetic components (right). The dashed vertical line indicates the $E_r$ predicted by the intrinsic and NBI rotation model. The electrons have a co-current neoclassical offset rotation and contribute a small co-current NTV torque density at the $E_r$ predicted by the rotation model.}
\end{figure}

In order to decouple the influence of the FI ripple and the TBM ripple, $\tau^{NTV}$ at $r/a = 0.9$ is calculated for toroidal modes (i) $\abs{n} \leq 18$, (ii) $\abs{n} = 18$, and (iii) $\abs{n} < 18$, shown in figure \ref{fig:Torque_comparingTBMandFI}. For $\abs{n} \leq 18$ and $\abs{n} = 18$, VMEC free boundary equilibria were computed including these toroidal modes. For $\abs{n} < 18$, the SFINCS calculation was performed to include the desired $n$ from the VMEC fields. Here $B$ was decomposed as,
\begin{gather}
B = \sum_{m,n} b_{mnc} \cos(m\theta-n\zeta) + b_{mns} \sin(m\theta-n\zeta),
\end{gather}
where $\theta$ and $\zeta$ are VMEC angles. The quantities $B^{\theta,\zeta}, B_{r,\theta,\zeta}, \sqrt{g}$, and position vectors $\partder{\bm{r}}{x^k}$ and $\nabla x^k$ are similarly decomposed such that the DKE can be solved for the desired toroidal modes.

We note that although the TBM ripple is largely an $\abs{n} = 1$ perturbation, there is a small contribution to the $\abs{n} = 18$ ripple. While the FIs decrease the magnitude of the $\abs{n} =18$ component of $B$, the TBM contributes most strongly to low mode numbers. As SFINCS is not linearized in the perturbing field, the torque due to both FIs and TBMs is not the sum of the torques due to these ripple fields considered individually. 

We find that the $\abs{n} = 18$ component drives about 100 times more torque than the other components of $B$. This result is in a agreement with most relevant rippled tokamak transport regimes feature positive scaling with $n$. For tokamak banana diffusion in the $\sqrt{\nu}$ regime\cite{Shaing2008} ion transport scales as $\Gamma_{\psi} \sim \sqrt{n}$ and in the $1/\nu$ regime\cite{Shaing2003} $\Gamma_{\psi} \sim n^2$.  In the collisionless detrapping-retrapping ripple transport regime,\cite{Shaing1982a,Shaing1982b} $\Gamma_{\psi} \propto G(\alpha)$ where $G(\alpha)$ increases with increasing $n$. Moreover, the low mode number ripple does not result in local ripple wells along a field line. This matches our findings that the higher harmonic $\abs{n} = 18$ ripple contributes more strongly to $\tau^{NTV}$ than the $\abs{n} < 18$ ripple. 

\begin{figure}[h!]
\centering
\includegraphics[width=0.7\textwidth]{Torque_comparingTBMandFI.png}
\caption{\label{fig:Torque_comparingTBMandFI} The NTV torque density at $r/a = 0.9$ for toroidal mode numbers (i) $\abs{n} = 18$ (purple solid), (ii) $\abs{n} < 18$ (brown dash dot), and (iii) $\abs{n} \leq 18$ (green dashed). The TBM ripple contributes most strongly to low $\abs{n}$, while the FIs and TF ripple contribute to $\abs{n} = 18$. The low $n$ TBM ripple does not contribute as strongly to the NTV torque density as the $\abs{n} = 18$ component of $\bm{B}$ does.}
\end{figure}

In figure \ref{fig:Torque_radiusscaling}, the SFINCS calculation of $\tau^{\text{NTV}}$ with TF ripple only is shown at $r/a$ = 0.5, 0.7, and 0.9. For these three radii the maximum $\delta_B = 0.26\%$,  0.51\%, and 0.82\% respectively. As $\tau^{\text{NTV}}$ scales with $\delta_B$ in most rippled tokamak regimes, it is reasonable to expect that the magnitude of $\tau^{\text{NTV}}$ would decrease with decreasing radius. On the other hand, transport scales strongly with $T_i$. In the $\sqrt{\nu}$ banana diffusion regime\cite{Shaing2008} $\Gamma_{\psi} \sim v_{ti}^4 \sqrt{\nu_{ii}} \sim T_i^{5/4}$. The combined effect of decreased ripple and increased temperature with decreasing radius leads to comparable torques with decreasing radius in the presence of significant $E_r$.  The scaling with $T_i$ is even stronger in the $1/\nu$ banana diffusion regime\cite{Shaing2009_sbp}, where $\Gamma_{\psi} \sim v_{ti}^4/\nu_ii \sim T_i^{7/2}$. Indeed, we find that the magnitude of $\tau^{\text{NTV}}$ at $E_r = 0$ increases with decreasing radius.
 
\begin{figure}[h!]
\centering
\includegraphics[width=0.7\textwidth]
{Torque_radiusscaling.png}
\caption{\label{fig:Torque_radiusscaling} SFINCS calculation of NTV torque density ($\tau^{\text{NTV}}$) as a function of ion $\langle V_{||} B \rangle/\langle B^2 \rangle^{1/2}$ for VMEC geometry with TF ripple only at $r/a$ = 0.5 (blue solid), 0.7 (red dashed), and 0.9 (green dash-dot). Although the field ripple decreases with radius (maximum $\delta_B = 0.82\%$ at $r/a = 0.9$, $\delta_B = 0.51\%$ at $r/a = 0.7$, $\delta_B = 0.26\%$ at $r/a = 0.5$), transport near $E_r = 0$ increases with decreasing radius because of scaling of ripple transport with $T_i$.\cite{Shaing2003}}
\end{figure}

In figure \ref{fig:alltorque}, we compare the magnitude of $\tau^{\text{NTV}}$ with $\tau^{\text{NBI}}$ and $\tau^{\text{turb}} = -\nabla \cdot \Pi_{\text{int}}$, the turbulent momentum source causing intrinsic rotation. For the $\tau^{\text{NTV}}$ profile, the intrinsic rotation model and NBI rotation model are used to estimate $E_r$ at each radius. The quantity $\tau^{\text{NBI}}$ is determined from NUBEAM calculations,\cite{Poli2014} and $\tau^{\text{turb}}$ is estimated using $\Pi_{\text{int}} \sim (\rho_{\theta}/L_T) \widetilde{\Pi}(\nu_*) Q (\langle R \rangle/v_{ti})$ (see appendix \ref{turbQ}). At $r/a = 0.55$, $\tau^{\text{NTV}}$ with the NBI rotation model ($E_r$ = 0.03 kV/m) is about 20 times larger than $\tau^{\text{NTV}}$ with the intrinsic rotation model ($E_r$ = 46.9 kV/m), as the $E_r$ predicted by the NBI model is small enough that $1/\nu$ transport may dominate. Note that the turbulent torque produces much rotation in the pedestal according to this model as $\tau^{\text{turb}} \propto 1/L_T$. The integrated NTV torque with the turbulent rotation model, -40 Nm, is comparable to the NBI torque, 35 Nm, while the turbulent torque is significantly larger, 93 Nm. The integrated NTV torque with the NBI rotation model is -52 Nm.

In the region $0.5 \lesssim r/a \lesssim 0.9$, NTV torque may dominate the rotation profile and will likely significantly damp rotation, decreasing MHD stability. However, the resulting rotation profile may be sheared because of the significant counter-current NTV source at the edge and co-current NBI source in the core. This may provide a rotation shear as high as $\Delta V_{\zeta}/ \Delta r \approx 0.8 (v_{ti}/R)$ near $r/a = 0.5$, possibly large enough to suppress microturbulence\cite{Hahm1994} and promote the formation of an internal transport barrier. The rotation shear may be even more significant if the rotation profile is similar to that predicted by the NBI rotation model because of the magnified $\tau^{NTV}$ near $r/a = 0.55$ where $E_r \approx 0$. 

\begin{figure}[h!]
\centering
\includegraphics[width=0.7\textwidth]{AllTorquePlot.png}
\caption{\label{fig:alltorque} Profiles of NTV torque density ($\tau^{\text{NTV}}$) calculated with SFINCS, NBI torque density calculated from NUBEAM ($\tau^{\text{NBI}}$), and estimate of turbulent intrinsic rotation momentum source ($\tau^{\text{turb}}$). The quantity $\tau^{\text{NTV}}$ is calculated using $E_r$ determined by the intrinsic rotation model and NBI rotation model described in section \ref{rotation}. Turbulent torque is estimated using $\tau^{\text{turb}} \sim -\Pi_{\text{int}}/a$ where $\Pi_{\text{int}} \sim \rho_{*, \theta} \widetilde{\Pi}(\nu_*) Q \langle R \rangle/v_{ti}$ (see appendix \ref{turbQ} for details).}
\end{figure}

\FloatBarrier

\section{Scaling with Ripple Magnitude}\label{scaling}
The scaling of NTV transport with the magnitude of $\delta_B$ shows some agreement with that predicted by Shaing for the $\sqrt{\nu}$ and $\nu$ regimes.\cite{Shaing2008, Shaing2009} In figure \ref{fig:scalescan}, the NTV torque density calculated by SFINCS is shown as a function of the magnitude of the ripple, $\delta_B$, for TF only geometry. The additional ferromagnetic ripple is not included, while the $\abs{n}=18$ components of $\bm{B}$, $\sqrt{g}$, and unit vectors are rescaled as described above. The quantity $\tau^{\text{NTV}}$ is calculated at $r/a = 0.9$ with $E_r = 30$ kV/m, corresponding to the intrinsic rotation estimate. The color-shaded background indicates the approximate regions of applicability of the $\sqrt{\nu}$ and $\nu$ banana diffusion regimes. The $1/\nu$ regime\cite{Shaing2003} does not apply at this $E_r$, as $\omega_E \gg \nu/\epsilon$ where $\omega_E = E_r/B^{\theta}$ is the $E\times B$ precession frequency. The radial electric field is also large enough that the resonance between $\bm{v}_{E}$ and $\bm{v}_{\text{m}}$ cannot occur, so the superbanana-plateau \cite{Shaing2009_sbp} and superbanana \cite{Shaing2009_sb} regimes are avoided. This significant$E_r$ may allow bounce-harmonic resonance to occur. The $l =1$ resonance condition $\omega_b - n(\omega_E + \omega_B) = 0$, will be satisfied for $v_{||} \approx v_{ti}$ with $E_r \sim 7$ kV/m at $r/a = 0.9$, where $\omega_b$ is the bounce frequency and $\omega_B$ is the toroidal magnetic drift precession.\cite{Park2009} However, we see no evidence of peaking or $\tau^{NTV}$ with $E_r$ that would be indicative of a bounce-harmonic resonance.

The banana diffusion $\sqrt{\nu}$ regime becomes relevant when the poloidal $\bm{E} \times \bm{B}$ precession frequency is larger than the effective collision frequency of detrapping banana particles, $\nu/\epsilon < \omega_E$, and the ripple magnitude is not large enough for collisionless detrapping to take effect, $\delta_B < \left(  \epsilon \nu/\omega_E \right)^{1/2}$. In the collisionless detrapping-retrapping regime, neoclassical fluxes scale with $\nu$.\cite{Shaing2009} This regime becomes relevant when $\delta_B > \left(  \epsilon \nu/\omega_E \right)^{1/2}$ and the perturbing field becomes large enough that poloidally trapped particles can become detrapped and retrapped by the ripple. In the $\sqrt{\nu}$ regime $\tau^{\text{NTV}} \sim \delta_B^2$ and in the $\nu$ regime $\tau^{\text{NTV}} \sim \delta_B$. The $1/\nu$ ripple trapping regime predicts $\Gamma_{\psi} \sim (\delta_B^{3/2})$.\cite{Stringer1972,Connor1973} Stellarator ripple transport in the $\sqrt{\nu}$ regime also predicts $\Gamma_{\psi}\sim(\delta_B^{3/2})$.\cite{Ho1987} For $\delta_B$ smaller than $\delta_B^* = 0.82\%$, the actual value of ripple at $r/a=0.9$ for ITER geometry, the scaling of $\tau^{\text{NTV}}$ with $\delta_B$ is slightly shallower than what is predicted. The departure of the SFINCS calculations from the quasilinear prediction, $\Gamma_{\psi} \propto \delta_B^2$, indicates the presence of nonlinear effects such as local ripple trapping. The departure from quasilinear scaling increases with $\delta_B$, which is consistent with comparisons of SFINCS with quasilinear NEO-2.\cite{Martitsch2016} 

\begin{figure}[h!]
\centering
\includegraphics[width=0.7\textwidth]
{scalescan.png}
\caption{\label{fig:scalescan} SFINCS calculations of NTV torque density as a function of $\delta_B$ at $r/a = 0.9$. A single value of $E_r = 30$ kV/m is used corresponding to the intrinsic rotation estimate. Color-shaded area indicates the approximate regions of applicability for rippled tokamak banana diffusion regimes described by Shaing: the $\sqrt{\nu}$ regime where $\tau^{\text{NTV}} \sim \delta_B^2$\cite{Shaing2008} and the $\nu$ regime where $\tau^{\text{NTV}} \sim \delta_B$.\cite{Shaing2009}}
\end{figure} 

\FloatBarrier

\section{Heat Flux Calculation}\label{heatflux}
As well as driving non-ambipolar particle fluxes, the breaking of toroidal symmetry drives an additional neoclassical heat flux. In figure \ref{fig:HeatFlux}, the SFINCS calculation of heat flux is shown for three magnetic geometries: (i) axisymmetric (blue solid), (ii) with TF ripple only (green dashed), and (iii) TF ripple with TBMs and FIs (red dash-dot). In the presence of TF ripple, the ripple drives an additional heat flux that is comparable to the axisymmetric heat flux. However, with the addition of the FIs the heat flux is reduced to the magnitude of the axisymmetric value, except near $E_r = 0$ where $1/\nu$ transport dominates. 

While the radial ripple-drive particle fluxes will significantly alter the ITER angular momentum transport, the neoclassical heat fluxes are insignificant in comparison to the turbulent heat flux. Note that the neoclassical heat flux is $\lesssim 5\%$  of the heat flux calculated from heating and fusion rate profiles (see appendix \ref{turbQ}), $Q\approx 0.2$ MW/m$^2$. Thus we can attribute $\gtrsim 95\%$ of the heat transport to turbulence. If ITER ripple were scaled up to $\delta_B \gtrsim 30\%$, the neoclassical ripple heat transport would be comparable to the anomalous transport.

\begin{figure}[h!]
\centering
\includegraphics[width=0.7\textwidth]
{HeatFlux.png}
\caption{\label{fig:HeatFlux} SFINCS calculation of neoclassical heat flux at $r/a = 0.9$ for three magnetic geometries: (i) axisymmetric (blue solid), (ii) with TF ripple only (green dashed), and (iii) TF ripple with TBMs and FIs (red dash-dot). The vertical dashed line corresponds to the intrinsic and NBI rotation estimate for $E_r$. These heat fluxes are much smaller than the anomalous heat transport is likely to be, $Q\approx 0.2$ MW/m$^2$,}
\end{figure}

\FloatBarrier

\section{Tangential Magnetic Drifts}\label{mds}
Although $(\bm{v}_E + \bm{v}_{\text{m}}) \cdot \nabla f_1$ is formally of lower order than the other terms in eq. \ref{kineticequation}, it has been found to be important when $\rho_* \sim \nu_*$ \cite{Calvo2016, Matsuoka2015} and has been included in other calculations of 3D neoclassical transport. In the SFINCS calculations shown in sections \ref{Erandv}, \ref{torque}, \ref{scaling}, and \ref{heatflux}, $\bm{v}_{\text{m}} \cdot \nabla f_1$ has not been included. As SFINCS does not maintain radial coupling of $f_1$, the poloidal and toroidal drifts are retained in this term while the radial drift is not. Note that the radial magnetic drift is retained in $\bm{v}_{\text{m}} \cdot \nabla f_0$. 

Several issues must be considered when the poloidal and toroidal magnetic drifts are present. A coordinate-dependence can be introduced when the $\bm{v}_{\text{m}} \cdot \nabla \psi$ term is ignored, as $\nabla \theta$ and $\nabla \zeta$ do not necessarily lie on the flux surface. For a coordinate-independent form, one must project $\bm{v}_{\text{m}}$ onto the flux surface. Additionally, when poloidal and toroidal drifts are retained, the effective particle trajectories do not necessarily conserve $\mu$ when $\mu = 0$. The drifts can be regularized in order to satisfy $\dot{\xi} (\xi = \pm 1) = 0$ where $\xi = v_{||}/v$. Regularization also eliminates the need for additional particle and heat sources due to the radially local assumption and preserves ambipolarity of axisymmetric systems.\cite{Sugama2016} To this end, we employ a coordinate-independent magnetic drift perpendicular to $\nabla \psi$,
\begin{gather}
\bm{v}_{\text{m}a}^{\perp} = \frac{v^2}{2B^2 \Omega_a} \frac{(\bm{B} \times \nabla \psi)}{\rvert \nabla \psi \rvert^2} \nabla \psi \cdot\left[(1-\xi^2)\nabla B + 2B\xi^2 (\bm{b} \cdot \nabla \bm{b}) \right].
\end{gather}
We note that the $\nabla B$ drift term is regularized while the curvature drift term is not. Note that the trapped portion of velocity space is most strongly affected by the addition of $\bm{v}_m \cdot \nabla f_1$ for which $\xi^2 \ll 1$. For this reason we can drop the curvature drift for regularization,
\begin{gather}
\bm{v}_{\text{m}a}^{\perp} = \frac{v^2}{2B^2 \Omega_a} (\bm{B} \times \nabla \psi) (1 - \xi^2) \frac{(\nabla \psi \cdot \nabla B)}{\rvert \nabla \psi \rvert^2}.
\label{eq:mds5}
\end{gather}
Note that this is similar to the form presented by Sugama,\cite{Sugama2016} but we have chosen a different form of regularization. We compare this form with the form of magnetic drifts without projection or regularization,
\begin{gather}
\bm{v}_{\text{m}a} = \frac{v^2}{2 \Omega_a B^2} (1 + \xi^2) \bm{B} \times \nabla B + \frac{v^2}{\Omega_a B} \xi^2 \nabla \times \bm{B}.
\label{eq:mds1}
\end{gather}

An $E_r$ scan at $r/a = 0.7$, where $\rho_*$ becomes comparable to $\nu_*$, is shown in figure \ref{fig:driftschemes}. When $\bm{v}_{\text{m}} \cdot \nabla f_1$ is added to the kinetic equation, the typical peak at $E_r = 0$ is shifted toward a slightly negative $E_r$, corresponding to the region where $(\bm{v}_E + \bm{v}_\text{M})\cdot \nabla \zeta \approx 0$. This corresponds to the superbanana-plateau transport regime in which the bounce averaged toroidal drift vanishes. Here the collisionality is large enough that superbananas cannot complete their collisionless trajectories but small enough that non-resonant trapped particles precess, $\nu_*^{\text{SB}} \ll \nu_* \ll \nu_*^{\text{SBP}}$, where $\nu_*^{\text{SBP}} = \rho_*q^2/\epsilon^{1/2}$ and $\nu_*^{SB} = \rho_* \delta_B^{3/2} q^2/\epsilon^2$.\cite{Shaing2009_sb, Shaing2009_sbp} 

We see that when the in-surface magnetic drifts are present the depth of the resonant peak is diminished. In the absence of magnetic drifts the superbanana-plateau resonance can only occur at $E_r = 0$. However, the resonance can occur for all particles regardless of pitch angle and energy. When tangential drifts are added to the DKE, the resonant peak will occur at the $E_r$ for which thermal trapped particles satisfy the resonance condition. However, only particles above a certain energy and at the resonant pitch angle will participate in the superbanana-plateau transport, thus the depth of the peak is diminished. Note that local ripple trapping might also contribute to the $1/\nu$ transport at small $E_r$. For $\abs{E_r} > 20$ kV/m, the range relevant for ITER, the addition of $\bm{v}_{\text{M}}\cdot \nabla f_1$ has a negligible effect on $\tau^{NTV}$. The addition of tangential drifts would not dramatically change the results in previous sections.  

\begin{figure}[h!]
\centering
\includegraphics[width=0.7\textwidth]{mdscomparison.png}
\caption{\label{fig:driftschemes} Calculation of NTV torque density, $\tau^{NTV}$, as a function of $E_r$ at $r/a = 0.7$. The orange dash-dot curve corresponds to a SFINCS calculation without $\bm{v}_m \cdot \nabla f_1$ in the DKE. The green solid curve corresponds to the addition of $\bm{v}_m \cdot \nabla f_1$ as given in eq. \ref{eq:mds1}. The blue dashed curve corresponds to the addition of the projected and regularized drift as given in eq. \ref{eq:mds5}.}
\end{figure}

\FloatBarrier

\section{Summary}\label{summary}

We calculate neoclassical transport in the presence of 3D magnetic fields, including toroidal field ripple and ferromagnetic components, for an ITER steady state scenario. We use an intrinsic turbulent rotation model described to estimate $E_r$ for neoclassical calculations. We find that even without $\tau^{\text{NTV}}$, toroidal rotation will be $\lesssim 2\% \,M_A$, which is likely not large enough to suppress resistive wall modes.\cite{Liu2004} We use VMEC free boundary equilibria in the presence of ripple fields to calculate neoclassical particle and heat fluxes using the drift-kinetic solver, SFINCS. At large radii $r/a \gtrsim 0.5$, $\tau^{\text{NTV}}$ is comparable to $\tau^{\text{NBI}}$ in magnitude but opposite in sign, which may result in flow damping at the edge and a decrease in MHD stability. The integral NTV torque, -40 Nm, is larger in magnitude than NBI torque, 35 Nm, so non-resonant magnetic braking cannot be ignored in analysis of ITER rotation. The torque profile may also result in a significant rotation shear which could suppress turbulent transport. While the addition of FIs significantly reduces the transport ($\approx 80\%$ reduction at $r/a = 0.9$), the low $n$ of the TBM ripple produce very little NTV torque. While the NTV torque has been shown to be important for ITER angular momentum balance, iteratively solving for the rotation profile with $\tau^{\text{NTV}}$ will be left for future consideration.

\appendix

\section{Parallel Flow is Less Sensitive to Perturbing Field Than Particle Fluxes} \label{parallelflow}
In this section we will show that the contribution to $V_{||}$ from $f_1$ is of order $\nu_* \ll 1$ smaller than the contribution to $\Gamma_{\psi}$. As $V_{||}$ is an odd moment of $v_{||}$ (eq. \ref{eq:parallelflow}), only the component of $f_1$ that is odd in $v_{||}$ will contribute. Similarly, only the component of $f_1$ that is even in $v_{||}$ will contribute to $\Gamma_{\psi}$ as it is an even moment of $v_{||}$ (eq. \ref{eq:particleflux}). We start with the following DKE,
\begin{gather}
v_{||} \bm{b} \cdot \nabla f_1 + \bm{v}_{\text{m}} \cdot \nabla \psi \partder{f_0}{\psi} = C(f_1),
\end{gather}
and perform a secondary expansion of $f_1 = f_1^0 + f_1^1 + ...$ in $\nu_* = \nu_{ii} Rq/(\epsilon^{3/2} v_{ti}) \ll 1$. We use field-aligned coordinates $(\psi, \theta, \zeta_0)$ where $\zeta_0 = q \theta - \zeta$ and velocity space coordinates $(v, \mu, \sigma)$ where $\sigma = v_{||}/\abs{v_{||}}$. To lowest order we have
\begin{gather}
v_{||} \bm{b} \cdot \nabla f_1^0 = 0.
\end{gather}
\label{firstorder}
This implies that $f_1^0(\psi, \zeta_0, v, \mu, \sigma)$ in the trapped region and $f_1^0(\psi, \zeta_0, v, \mu, \sigma)$ in the passing region. In the trapped region we must have that $f_{1,t}^0(\sigma = 1, \theta_b) = f_{1,t}^0(\sigma = -1, \theta_b)$ where $\theta_b$ is the bounce angle and $\sigma = v_{||}/\abs{v_{||}}$. Therefore, $f_{1,t}^0(\sigma = 1) = f_{1,t}^0(\sigma = -1)$, and $f_{1,t}^0$ is even in $v_{||}$. We next consider the parity of $f_{1,p}^0$ in the passing region of velocity space. The next order equation in $\nu_*$ is,
\begin{gather}
v_{||} \bm{b} \cdot \nabla f_1^1 + \bm{v}_{\text{m}} \cdot \nabla \psi \partder{f_0}{\psi} = C(f_1^0).
\label{secondorder}
\end{gather}
We apply the transit averaging operation, $\langle ... \rangle_t$, to eq. \ref{secondorder},
\begin{gather}
\langle ... \rangle_t = \int_0^{2\pi} \frac{d \theta \, B}{v_{||} B^{\theta}} (...),
\end{gather}
which annihilates the left hand side. Let $C$ be a pitch angle scattering operator,
\begin{gather}
C = \frac{\nu v_{||}}{B} \partder{}{\mu} \left( v_{||} \mu \partder{}{\mu} \right).
\end{gather}
Performing indefinite integration in $\mu$, we can rewrite eq. \ref{secondorder} as,
\begin{gather}
f_{1,p}^0(v, \sigma, \mu, \zeta_0) = A(v, \sigma, \zeta_0) + \frac{B(v, \sigma, \zeta_0)}{\int_0^{2\pi} \frac{d \theta v_{||}}{B^{\theta}}} \log(\mu)
\label{passing}
\end{gather}
For some functions $A$ and $B$. Because of the divergence at $\mu=0$ in the second term in eq. \ref{passing}, we must have $B = 0$. As $f_1^0$ must be continuous across the trapped-passing boundary, $f_{1,p}^0$ cannot be a function of $\sigma$ and must also have even parity in $v_{||}$. We can solve eq. \ref{secondorder} for $f_1^1$ by performing indefinite integration in $\theta$. 
\begin{gather}
f_{1}^1 = \int \frac{d \theta B}{B^{\theta} v_{||}} \left[C(f_1^0) - \bm{v}_{\text{m}} \cdot \nabla \psi \partder{f_0}{\psi} \right] + D(\psi, \zeta_0, v, \sigma, \mu),
\end{gather}
for some function $D$. Within the integrand, $C(f_1^0)$ can be rewritten by expanding $f_1^0$ in even Legendre polynomials in $\xi$, $f_1^0 = \sum_{\text{even }l} f_{1,l}^0 P_l(\xi)$,
\begin{gather}
C(f_1^0) = \sum_{\text{even }l} l(l+1) f_{1,l}^0 P_l(\xi).
\end{gather}
So $C(f_1^0)$ will have the same parity as $f_1^0$. As both $C(f_1^0)$ and $\bm{v}_{\text{m}} \cdot \nabla \psi$ are even in $v_{||}$, $f_1^1$ will have a component odd in parity and will contribute to $V_{||}$. Therefore, the departure of $V_{||}$ from the axisymmetric value will occur at order $\nu_*$ higher than the departure of $\Gamma_{\psi}$ from its axisymmetric value.

\section{Approximate Turbulent Heat Flux and Torque}\label{turbQ}

As $\Pi_{\text{int}}$ is proportional to $Q$ in our model, we must estimate $Q$ using the input heating power and D-T fusion rates calculated with TRANSP and TSC. The LH, NBI, and ECH power densities ($P_{\text{LH}}$, $P_{\text{NBI}}$, and $P_{\text{ECH}}$) are integrated along with the fusion reaction rate density ($R_{\text{DT}}$) to calculate the total integrated heating source, $H(r)$,
\begin{gather}
\int_0^r dV \, H(r) = \int_0^r dV \left(\, P_{\text{LH}} + P_{\text{NBI}} + P_{\text{ECH}} + R_{\text{DT}} (3.5 \text{MeV}) \right).
\end{gather}
As $\int Q dS = \int H dV$, 
\begin{gather}
Q(r) = \frac{\int_0^{V(r)} dV' \left(\, P_{\text{LH}} + P_{\text{NBI}} + P_{\text{ECH}} + R_{\text{DT}} (3.5 \text{MeV}) \right)}{A(r)},
\end{gather}
where $A(r) = V'(r)$ is the flux surface area. We have shown in section \ref{heatflux} that the neoclassical heat flux is insignificant in comparison to $Q(r)$, so we can attribute $Q(r)$ to turbulent heat transport. The calculated $Q$ is shown in figure \ref{fig:turbHeatFlux}.

\begin{figure}[h!]
\centering
\includegraphics[width=0.7\textwidth]{turbHeatFlux.png}
\caption{\label{fig:turbHeatFlux} Heat flux $Q$ calculated with input heating and fusion rate profiles from TRANSP and TSC.}
\end{figure}

We estimate $\tau^{\text{turb}} = - \nabla \cdot \Pi_{\text{int}} \sim -\Pi_{\text{int}}/a$ using 
\begin{gather}
\Pi_{\text{int}} \sim \frac{\rho_{\theta} \widetilde{\Pi}(\nu_*) Q \langle R \rangle}{v_{ti} L_T}.
\end{gather}
The quantity $\tau^{\text{turb}}$ is shown in figure \ref{fig:alltorque}.

\FloatBarrier

\section*{Acknowledgements}
The authors would like to thank F. Parra, J. Hillesheim, J. Lee, G. Papp, S. Satake and J. Harris for helpful input and discussions. This work was supported by the US Department of Energy through grants DE-FG02-93ER-54197 and DE-FC02-08ER-54964. The computations presented in this paper have used resources at the National Energy Research Scientific Computing Center (NERSC). 

\bibliography{ITERNTV}

\end{document}
